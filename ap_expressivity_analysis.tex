\section{Proofs of Theoretic Results in Section~\ref{sec:simplical_representation}}

Proof of Lemma~\ref{lem:union_cohesive}:
\begin{proof}
    By assumption, $S_1$ and $S_2$ are $c$-cohesive sets with $|S_1 \cap S_2| \geq c$. We need to show that $S_1 \cup S_2$ is $c$-cohesive, i.e., for any distinct vertices $u, v \in S_1 \cup S_2$, we have $\kappa_G(u,v) \geq c$.

    Consider any distinct vertices $u, v \in S_1 \cup S_2$. There are three cases:
    \begin{enumerate}[leftmargin=*,nosep]
        \item If both $u, v \in S_1$, then $\kappa_G(u,v) \geq c$ since $S_1$ is $c$-cohesive.
        \item If both $u, v \in S_2$, then $\kappa_G(u,v) \geq c$ since $S_2$ is $c$-cohesive.
        \item If $u \in S_1 \setminus S_2$ and $v \in S_2 \setminus S_1$ (or vice versa), we proceed as follows:
    \end{enumerate}

    For the third case, suppose $u \in S_1 \setminus S_2$ and $v \in S_2 \setminus S_1$. Let $R \subseteq V(G) \setminus \{u,v\}$ be any set with $|R| < c$. We will show that $u$ and $v$ remain connected in $G - R$.

    Since $|S_1 \cap S_2| \geq c > |R|$, there exists at least one vertex $t \in (S_1 \cap S_2) \setminus R$. 

    Since $S_1$ is $c$-cohesive and $|R| < c$, vertices $u$ and $t$ remain connected in $G - R$ (by definition of $c$-cohesiveness). Similarly, since $S_2$ is $c$-cohesive, vertices $t$ and $v$ remain connected in $G - R$.

    Therefore, there is a path from $u$ to $t$ in $G - R$ and a path from $t$ to $v$ in $G - R$, which together form a path from $u$ to $v$ in $G - R$.

    This shows that for any $u \in S_1 \setminus S_2$ and $v \in S_2 \setminus S_1$, we have $\kappa_G(u,v) \geq c$ (since no set $R$ with $|R| < c$ can separate $u$ and $v$).

    Therefore, $S_1 \cup S_2$ is $c$-cohesive.
\end{proof}


Proof of Lemma~\ref{lem:intersection-separator}:
\begin{proof}
    Since $\sigma$ and $\varrho$ are maximal in $\Delta(G)$, they are minimal $c$-cohesive sets for some integers $c_\sigma, c_\varrho \geq 1$, respectively. By Remark~\ref{rem:cohesive_monotonicity}, any non-empty face of a $c$-cohesive set is itself $c'$-cohesive for some $c' \leq c$; in particular, $\tau$ is non-empty (otherwise $\sigma \cup \varrho$ would be cohesive by Lemma~\ref{lem:union_cohesive}, contradicting maximality).

    Suppose, for contradiction, that there exists a path $P$ in $G$ from a vertex $u \in \sigma \setminus \tau$ to a vertex $v \in \varrho \setminus \tau$ that avoids $\tau$. Let $c = |\tau| - 1$. Because $\sigma$ is a minimal $c_\sigma$-cohesive set, we have $|\sigma| = c_\sigma + 1$, and similarly $|\varrho| = c_\varrho + 1$. Moreover, since $\tau \subsetneq \sigma$ and $\tau \subsetneq \varrho$, it follows that $|\tau| \leq \min\{c_\sigma, c_\varrho\}$.

    The existence of $P$ together with the $c_\sigma$- and $c_\varrho$-cohesiveness of $\sigma$ and $\varrho$ implies that every pair of vertices in $\sigma \cup \varrho$ is linked by at least $\min\{c_\sigma, c_\varrho\}$ internally vertex-disjoint paths. In particular, if $|\tau| = \min\{c_\sigma, c_\varrho\}$, then $\sigma \cup \varrho$ satisfies the definition of a $\min\{c_\sigma, c_\varrho\}$-cohesive set. This contradicts the maximality of $\sigma$ and $\varrho$ in $\Delta(G)$, as their union would contain a larger cohesive set.

    Hence, no such path $P$ exists, and $\tau$ separates $\sigma \setminus \tau$ from $\varrho \setminus \tau$.

    To establish minimality, let $R \subsetneq \tau$. Since $\sigma$ is cohesive, every vertex in $\sigma \setminus \tau$ remains connected to every vertex in $\tau \setminus R$ in $G - R$; an analogous statement holds for $\varrho$. Because $\tau \setminus R \neq \emptyset$, there exists a vertex $t \in \tau \setminus R$ that connects $\sigma \setminus \tau$ to $\varrho \setminus \tau$ in $G - R$. Thus, $R$ does not separate the two sides, proving that $\tau$ is a \emph{minimal} vertex separator.
\end{proof}

Proof of Lemma~\ref{lem:maximal_simplex_network_invariant}:
\begin{proof}
    Let $G$ and $H$ be isomorphic graphs with isomorphism $\pi: V(G) \to V(H)$. 
    We show that the maximal simplex networks of $G$ and $H$ are isomorphic.
    
    First, observe that $\pi$ preserves the cohesive structure: a set $S \subseteq V(G)$ is $c$-cohesive in $G$ if and only if $\pi(S) \subseteq V(H)$ is $c$-cohesive in $H$. 
    This follows since $\kappa_G(u,v) = \kappa_H(\pi(u),\pi(v))$ for all $u,v \in V(G)$.
    
    Consequently, $\pi$ maps minimal $c$-cohesive sets in $G$ to minimal $c$-cohesive sets in $H$, preserving the simplicial complex structure. 
    In particular, maximal simplices in $\Delta(G)$ correspond bijectively to maximal simplices in $\Delta(H)$ under $\pi$.
    
    Similarly, minimal separators in $G$ correspond to minimal separators in $H$: if $S$ separates vertices $u,v$ in $G$, then $\pi(S)$ separates $\pi(u),\pi(v)$ in $H$, and $S$ is minimal if and only if $\pi(S)$ is minimal.
    
    The incidence relations are preserved: if maximal simplices $\sigma,\varrho$ in $\Delta(G)$ intersect in separator $\tau$, then the corresponding maximal simplices $\pi(\sigma),\pi(\varrho)$ in $\Delta(H)$ intersect in separator $\pi(\tau)$. 
    
    Therefore, $\pi$ induces an isomorphism between the maximal simplex networks of $G$ and $H$.
\end{proof}

Proof of Lemma~\ref{lem:maximal_simplex_network_acyclic}:
\begin{proof}
    Removing any separator disconnects graph $G$. Thus removing any separator node disconnects the maximal simplex network.
    Removing any maximal simplex would also remove separators contained in it. Thus removing any simplex node disconnects the maximal simplex network.
    Consequently, the maximal simplex network is acyclic.
\end{proof}


Proof of Lemma~\ref{lem:subgraph-subnetwork-correspondence}:
\begin{proof}
Let $Q \subseteq G$ be a connected subgraph. Define $\Delta_Q(G)$ as the subcomplex of $\Delta(G)$ consisting of all minimal $c$-cohesive sets $\sigma$ such that $\sigma \subseteq V(Q)$ for some $c \geq 0$.

\noindent\textbf{Step 1: $\Delta_Q(G)$ is a well-defined subcomplex.} 
Since every minimal $c$-cohesive set in $\Delta_Q(G)$ is contained in $V(Q)$ and belongs to $\Delta(G)$, we verify the subcomplex property: for any $\sigma \in \Delta_Q(G)$ and any non-empty subset $\tau \subseteq \sigma$, we have $\tau \subseteq V(Q)$ and $\tau \in \Delta(G)$ by the hereditary property of abstract simplicial complexes. Therefore, $\tau \in \Delta_Q(G)$, confirming that $\Delta_Q(G)$ is closed under taking non-empty subsets.

\noindent\textbf{Step 2: $\Delta_Q(G)$ is connected.}
Let $\sigma_1, \sigma_2 \in \Delta_Q(G)$ be arbitrary minimal cohesive sets. Since $Q$ is connected, there exists a path $P$ in $Q$ connecting some vertex $u \in \sigma_1$ to some vertex $v \in \sigma_2$. 

We construct a sequence of minimal cohesive sets connecting $\sigma_1$ to $\sigma_2$ as follows: traversing $P$ from $u$ to $v$, let $w_0 = u, w_1, \ldots, w_\ell = v$ be the vertices of $P$ in order. For each consecutive pair $w_i, w_{i+1}$ on $P$, there exists a minimal $c$-cohesive set $\tau_i$ containing both vertices (since $w_i$ and $w_{i+1}$ are connected in $Q$ and hence belong to some common cohesive structure). 

Starting with $\tau_0 = \sigma_1$, we can find a sequence $\tau_0, \tau_1, \ldots, \tau_k$ where $\tau_k = \sigma_2$ and $\tau_i \cap \tau_{i+1} \neq \emptyset$ for all $i$. This sequence demonstrates path-connectedness in the 1-skeleton of $\Delta_Q(G)$, establishing connectivity of the subcomplex.

\noindent\textbf{Step 3: Subnetwork correspondence.}
In the maximal simplex network of $G$, each maximal simplex node represents a maximal simplex from $\Delta(G)$, and separator nodes represent minimal separators formed by intersections of maximal simplices.

The subcomplex $\Delta_Q(G)$ corresponds to the subset of maximal simplices from $\Delta(G)$ that are entirely contained in $V(Q)$. By Definition~\ref{def:maximal_simplex_network}, these maximal simplices become nodes in the induced subnetwork, while their pairwise intersections that lie within $V(Q)$ become the separator nodes connecting them.

The connectivity preservation follows from two key observations: (i) any two maximal simplices in $\Delta_Q(G)$ that intersect do so in a minimal separator within $V(Q)$, creating an edge in the subnetwork; (ii) the connectivity of $Q$ ensures that for any partition of $V(Q)$ by a minimal separator, the remaining components remain connected within $Q$. Therefore, the induced subnetwork maintains the same connectivity structure as $\Delta_Q(G)$.

Consequently, $\Delta_Q(G)$ forms a connected subcomplex whose structure is faithfully represented by a connected subnetwork of the maximal simplex network.
\end{proof}
