\section{Proofs of Theoretic Results in Section~\ref{sec:simplical_representation}}

Proof of Lemma~\ref{lem:union_cohesive}:
\begin{proof}
    By assumption, $S_1$ and $S_2$ are $c$-cohesive sets with $|S_1 \cap S_2| \geq c$. We need to show that $S_1 \cup S_2$ is $c$-cohesive, i.e., for any distinct vertices $u, v \in S_1 \cup S_2$, we have $\kappa_G(u,v) \geq c$.

    Consider any distinct vertices $u, v \in S_1 \cup S_2$. There are three cases:
    \begin{enumerate}[leftmargin=*,nosep]
        \item If both $u, v \in S_1$, then $\kappa_G(u,v) \geq c$ since $S_1$ is $c$-cohesive.
        \item If both $u, v \in S_2$, then $\kappa_G(u,v) \geq c$ since $S_2$ is $c$-cohesive.
        \item If $u \in S_1 \setminus S_2$ and $v \in S_2 \setminus S_1$ (or vice versa), we proceed as follows:
    \end{enumerate}

    For the third case, suppose $u \in S_1 \setminus S_2$ and $v \in S_2 \setminus S_1$. Let $R \subseteq V(G) \setminus \{u,v\}$ be any set with $|R| < c$. We will show that $u$ and $v$ remain connected in $G - R$.

    Since $|S_1 \cap S_2| \geq c > |R|$, there exists at least one vertex $t \in (S_1 \cap S_2) \setminus R$. 

    Since $S_1$ is $c$-cohesive and $|R| < c$, vertices $u$ and $t$ remain connected in $G - R$ (by definition of $c$-cohesiveness). Similarly, since $S_2$ is $c$-cohesive, vertices $t$ and $v$ remain connected in $G - R$.

    Therefore, there is a path from $u$ to $t$ in $G - R$ and a path from $t$ to $v$ in $G - R$, which together form a path from $u$ to $v$ in $G - R$.

    This shows that for any $u \in S_1 \setminus S_2$ and $v \in S_2 \setminus S_1$, we have $\kappa_G(u,v) \geq c$ (since no set $R$ with $|R| < c$ can separate $u$ and $v$).

    Therefore, $S_1 \cup S_2$ is $c$-cohesive.
\end{proof}


\begin{proof}
    \noindent($\Rightarrow$) 
    Assume $\varphi\colon V(G) \to V(H)$ is a graph isomorphism. 
    For any distinct $u,v \in V(G)$, the bijection $\varphi$ induces a one-to-one correspondence between vertex-disjoint $(u,v)$-paths in $G$ and vertex-disjoint $(\varphi(u),\varphi(v))$-paths in $H$. 
    By Menger's theorem, $\kappa_G(u,v) = \kappa_H(\varphi(u),\varphi(v))$.
    It follows that for any $S \subseteq V(G)$,
    \[
        S \text{ is (minimal) $c$-cohesive in } G \quad \Longleftrightarrow \quad \varphi(S) \text{ is (minimal) $c$-cohesive in } H.
    \]
    Hence $\varphi$ extends to a simplicial isomorphism $\Delta(G) \to \Delta(H)$. 
    Moreover, since $\varphi$ preserves adjacency, we have
    \[
        \ell_G(\{u,v\}) = \mathbb{1}[(u,v) \in E(G)] = \mathbb{1}[(\varphi(u),\varphi(v)) \in E(H)] = \ell_H(\{\varphi(u),\varphi(v)\}),
    \]
    yielding an isomorphism of labeled complexes $(\Delta(G),\ell_G) \cong (\Delta(H),\ell_H)$.

    \noindent($\Leftarrow$) 
    Conversely, suppose $\psi\colon V(G) \to V(H)$ is an isomorphism of labeled simplicial complexes $(\Delta(G),\ell_G) \cong (\Delta(H),\ell_H)$.
    Then $\psi$ is a vertex bijection preserving all simplices and satisfying
    \[
        \ell_G(\{u,v\}) = \ell_H(\{\psi(u),\psi(v)\}) \quad \forall u,v \in V(G), u \neq v.
    \]
    The edge set of $G$ can be recovered as
    \[
        E(G) = \bigl\{\{u,v\} : \ell_G(\{u,v\}) = 1\bigr\},
    \]
    and analogously for $H$.
    Therefore,
    \[
        \{u,v\} \in E(G) \quad \Longleftrightarrow \quad \ell_G(\{u,v\}) = 1 \quad \Longleftrightarrow \quad \ell_H(\{\psi(u),\psi(v)\}) = 1 \quad \Longleftrightarrow \quad \{\psi(u),\psi(v)\} \in E(H),
    \]
    so $\psi$ is a graph isomorphism $G \cong H$.
\end{proof}
