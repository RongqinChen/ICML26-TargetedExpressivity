\section{Preliminaries}
This section establishes notation and reviews core concepts from graph theory and combinatorial topology that form the foundation of our work.
Specifically, we require the notion of vertex connectivity in graphs and its generalization via the framework of abstract simplicial complexes.

\paragraph{Notations.} Let $\{\cdot\}$ denote a set, $\{\!\!\{ \cdot \}\!\!\}$ denote a multiset (a set that allows repetition), and $(\cdot)$ denote a tuple.
As usual, let $[n] = \{1,2,\ldots, n \}$. Let $G=(V(G),E(G), \ell_G)$ be an undirected, colored graph, where $V(G)=[n]$ is the vertex set with $n$ vertices, $E(G)\subseteq V(G)\times V(G)$ is the edge set, and $\ell_G\colon V(G) \to C$ is the graph coloring function with $C=\{c_1,\ldots,c_{nc}\}$ denote a set of $nc$ distinct colors.

Let $\spd(u, v)$ denote the shortest path distance between $u$ and $v$.
We use $x_{v} \in \mathbb{R}^{d_v}$ to denote attributes of vertex $v \in V(G)$ and $e_{uv}\in \mathbb{R}^{d_e}$ to denote attributes of edge $(u,v) \in E(G)$. They are usually the one-hot encoding of the vertex and edge color respectively. We say that two graphs $G$ and $H$ are \textit{isomorphic} (denoted as $G\simeq H$) if there exists a bijection $\pi \colon V(G) \to V(H)$ such that $\forall u, v \in V(G), (u, v) \in E(G) \Leftrightarrow (\pi(u), \pi(v)) \in E(H)$ and $\forall v \in V(G), \ell_G(v) = \ell_H(\pi(v)) $.



\paragraph{Graph Connectivity.}
Throughout, we work with finite, simple, undirected graphs. For a positive integer $n$, let $K_n$ denote the complete graph on $n$ vertices.

For a graph $G$, a \emph{(vertex) separator} is a vertex set $S \subseteq V(G)$ such that the induced subgraph $G[V(G) \setminus S]$ has more than one connected component. A separator is \emph{minimal} if no proper subset of $S$ is a separator.
The \emph{(vertex) connectivity} $\kappa(G)$ is defined as follows: $\kappa(K_n) = n-1$ for $n \ge 1$ (with the convention $K_1$ is $0$-connected), and for any other connected graph $G$, $\kappa(G)$ is the minimum size of a separator. A graph is \emph{$c$-connected} if $\kappa(G) \ge c$.

For distinct vertices $u, v \in V(G)$, the \emph{local connectivity} $\kappa_G(u,v)$ is the maximum number of internally vertex-disjoint $u$--$v$ paths in $G$. For non-adjacent pairs, Menger's Theorem provides the fundamental equivalence between this path-counting view and the separator-based view of connectivity.
\begin{theorem}[Menger’s Theorem (vertex version)~\cite{Menger1927,Diestel2017}]
    \label{thm:menger}
    Let $u$ and $v$ be distinct, non-adjacent vertices in a graph $G$. The minimum size of a vertex set $S \subseteq V(G)\setminus\{u,v\}$ that separates $u$ from $v$ equals $\kappa_G(u,v)$.
\end{theorem}
The global connectivity relates to local connectivity via $\kappa(G) = \min\{\kappa_G(u,v) : u, v \in V(G), u \neq v\}$ for non-complete graphs.

\paragraph{Abstract Simplicial Complexes.}
To extend the concept of connectivity beyond pairwise relations, we employ the language of abstract simplicial complexes.
An \emph{abstract simplicial complex} $\Delta$ on a finite vertex set $V$ is a collection of non-empty subsets of $V$ closed under inclusion: if $\sigma \in \Delta$ and $\emptyset \neq \tau \subseteq \sigma$, then $\tau \in \Delta$.
Elements of $\Delta$ are called \emph{simplices}.
A simplex $\sigma$ with $|\sigma| = d+1$ is a \emph{$d$-simplex}, and $\dim(\sigma) := d$. The \emph{dimension} of $\Delta$ is $\dim(\Delta) := \max\{\dim(\sigma) : \sigma \in \Delta\}$.
For $\tau \subseteq \sigma$, we say $\tau$ is a \emph{face} of $\sigma$ and $\sigma$ is a \emph{coface} of $\tau$.
If $\dim(\sigma)=\dim(\tau)+1$, $\tau$ is an immediate \emph{face} of $\sigma$ and $\sigma$ is an immediate \emph{coface} of $\tau$.
The \emph{$d$-skeleton} $\Delta^{(d)}$ is the subcomplex consisting of all simplices of dimension at most $d$. In particular, $\Delta^{(0)} = V$ and $\Delta^{(1)}$ is a simple graph on $V$. A complex $\Delta$ is \emph{connected} if its $1$-skeleton $\Delta^{(1)}$ is connected as a graph.



\paragraph{The Weisfeiler-Lehman Hierarchy.}
The $k$-dimensional Weisfeiler-Lehman ($k$-WL) algorithm is a color-refinement procedure for graph isomorphism testing.
For $k \ge 2$, the $k$-WL algorithm assigns colors to $k$-tuples $\vv \in V(G)^k$.
Given a $k$-tuple $\vv = (v_1,\ldots,v_k)$ and a vertex $t \in V(G)$,
let $\vv_{/i} = (v_1,\ldots,v_{i-1},v_{i+1},\ldots,v_k)$ denote the tuple obtained by removing the $i$-th element.
And let $\vv_{t/i} = (v_1,\ldots,v_{i-1},t,v_{i+1},\ldots,v_k)$ denote the tuple obtained by replacing the $i$-th coordinate with $t$.
We define the \emph{$i$-th neighborhood} of $\vv$ as $N_i(\vv) = \{\vv_{t/i} \mid t \in V(G)\}$ for $i \in [k]$.
The initial coloring $\mathcal{C}^{k,0}_{\vv}$ typically encodes the isomorphism type of the subgraph induced by the vertices in $\vv$.
At iteration $l$, the $k$-WL algorithm refines the coloring via
\begin{equation}
    \label{eq:kwl}
    \mathcal{C}^{k,l}_{\vv}
    = \Gamma \left(
    \mathcal{C}^{k,l{-}1}_{\vv},\
    \left(
    \multiset{\mathcal{C}^{k,l{-}1}_{\vu} \mid \vu \in N_i(\vv)} \mid i \in [k]
    \right)
    \right),
\end{equation}
where $\Gamma$ is an injective hash function applied uniformly to all tuples.

\paragraph{$k$-WL GNNs.}
A $k$-WL GNN adopts the message-passing paradigm of the $k$-WL algorithm, thereby achieving equivalent expressive power.
The initial vertex representation is set to $\mX^{k,0}_{\vv} = \mathcal{C}^{k,0}_{\vv}$.
At each layer $l$, for every $k$-tuple $\vv$, the network updates its representation $\mX^{k,l}_{\vv}$ as follows:
\begin{equation}
    \label{eq:kwl_gnns}
    \begin{split}
        \mH^{k,l}_{\vv_{/i}} & = \oplus\left(\multiset{\mX^{k,l-1}_{\vu} \mid \vu \in N_i(\vv)}\right),                       \\
        \mX^{k,l}_{\vv}      & = \Psi^{k,l}\left(\mX^{k,l-1}_{\vv},\mH^{k,l}_{\vv_{/1}}, \ldots, \mH^{k,l}_{\vv_{/k}}\right),
    \end{split}
\end{equation}
where $\Psi^{k,l}$ is a learnable update function and $\oplus$ denotes a permutation-invariant multiset aggregation function (e.g., sum or mean).
