\section{Preliminaries}

Throughout, we work with finite, simple, undirected graphs. For a positive integer $n$, let $K_n$ denote the complete graph on $n$ vertices.

\paragraph{Graph Connectivity and Local Connectivity.}
For a graph $G$, a \emph{separator} is a vertex set $S \subseteq V(G)$ whose removal disconnects $G$. It is \emph{minimal} if no proper subset of $S$ is a separator. 
The \emph{(vertex) connectivity} $\kappa(G)$ is defined by $\kappa(K_1) = 0$, $\kappa(K_n) = n-1$ for $n \ge 2$, and for every other connected graph $G$, $\kappa(G)$ is the size of a smallest separator. A graph is \emph{$c$-connected} if $\kappa(G) \ge c$.
For distinct vertices $u, v \in V(G)$, the \emph{local connectivity} $\kappa_G(u,v)$ is the maximum number of internally vertex-disjoint $u$--$v$ paths in $G$. When $u$ and $v$ are adjacent, the edge $(u,v)$ itself is such a path. This path-based definition is equivalent to the classical separator-based one for non-adjacent pairs, by Menger’s Theorem:
\begin{theorem}[Menger’s Theorem (vertex version)~\cite{Menger1927,Diestel2017}]
    \label{thm:menger}
    For distinct non-adjacent vertices $u$ and $v$ in a graph $G$, the size of a minimum $u$--$v$ separator equals the maximum number of internally vertex-disjoint $u$--$v$ paths.
\end{theorem}

\paragraph{Abstract Simplicial Complexes.}
An \emph{abstract simplicial complex} $\Delta$ on a finite vertex set $V$ is a collection of non-empty subsets of $V$ that is closed under taking non-empty subsets: if $\sigma \in \Delta$ and $\emptyset \neq \tau \subseteq \sigma$, then $\tau \in \Delta$. Elements of $\Delta$ are called \emph{simplices}. A simplex $\sigma$ with $|\sigma| = d+1$ is a \emph{$d$-simplex}, and we set $\dim(\sigma) := d$. The dimension of $\Delta$ is $\dim(\Delta) := \max\{\dim(\sigma) : \sigma \in \Delta\}$.
If $\tau \subseteq \sigma$ with $\tau, \sigma \in \Delta$, then $\tau$ is a \emph{face} of $\sigma$ and $\sigma$ is a \emph{coface} of $\tau$. The \emph{$d$-skeleton} $\Delta^{(d)}$ consists of all simplices in $\Delta$ of dimension at most $d$. In particular, $\Delta^{(0)} = V$ and $\Delta^{(1)}$ is a simple graph on $V$. We say that $\Delta$ is \emph{connected} if its $1$-skeleton $\Delta^{(1)}$ is connected as a graph.
