\section{Analyzing GNN Expressivity}

\subsection{Complete Expressivity of $k$-FWL GNNs on Graphs without $(k{+}1)$-Cohesive Sets}

Let $\mathcal{G}_{k}$ denote the class of graphs that do not contain any $(k{+}1)$-cohesive sets. We establish that $k$-FWL GIN achieves complete expressivity over $\mathcal{G}_{k}$ with respect to both graph isomorphism discrimination and subgraph counting. This expressivity is formalized in the following two theorems.

\begin{theorem}[Isomorphism Discrimination]
\label{thm:isomorphism-discrimination}
Let $G_1, G_2 \in \mathcal{G}_{k}$. If $G_1 \not\cong G_2$, then there exists a $k$-FWL GIN that distinguishes $G_1$ from $G_2$.
\end{theorem}

\begin{theorem}[Subgraph Counting Expressivity]
\label{thm:subgraph-expressivity}
For any pattern graph $Q$, any graph $G \in \mathcal{G}_{k}$, and any vertex $v \in V(G)$, there exists a $k$-FWL GIN that computes a vertex representation $\mathbf{z}^{(k)}_{v}$ from which the local subgraph count can be uniquely determined. Formally, there exists a function $f: \mathbb{R}^d \to \mathbb{N}$ such that 
\[
f(\mathbf{z}^{(k)}_{v}) = \mathrm{iso}(Q, G, v),
\]
where $\mathrm{iso}(Q, G, v)$ denotes the number of subgraphs of $G$ isomorphic to $Q$ that contain vertex $v$.
\end{theorem}

The proofs of these theorems rely on the following structural correspondence lemma.

\begin{lemma}[Bijection between Maximal Simplex Networks and Simplicial $k$-Trees]
\label{lem:maximal-simplex-network-simplicial-k-trees}
Let $G \in \mathcal{G}_{k}$. Then there exists a bijection between the maximal simplex networks of $G$ and its simplicial $k$-tree set. More precisely:
Given the maximal simplex network of $G$, one can uniquely construct the simplicial $k$-tree set of $G$.
Conversely, given the simplicial $k$-tree set of $G$, one can uniquely construct its maximal simplex network.
\end{lemma}

As a result, we have
\begin{corollary}
    For two graphs $G_1, G_2 \in \mathcal{G}_{k}$, if $G_1 \not\cong G_2$, then their simplicial $k$-tree sets are different.
\end{corollary}

\subsection{Bottleneck: Fundamental Expressivity Limits from $(k{+}1)$-Cohesive Sets}

We prove this by constructing a family of graphs that all have the same simplicial $k$-tree set, but different $(k{+}1)$-cohesive sets.
