\section{Analyzing GNN Expressivity}

\subsection{Unfolding the Computational Trace of $k$-WL}
\label{sec:k_fwl_subtrees}

To characterize the expressivity of $k$-WL GNNs, we formalize \textit{$k$-WL-subtrees}---tree-structured computational histories that generalize WL-subtrees~\citep{xu2019powerful} from vertices to $k$-tuples.
While WL-subtrees capture neighborhood aggregation for MPNNs, $k$-WL-subtrees encode the hierarchical message-passing trajectories of $k$-WL over $k$-tuples.

\begin{definition}[WL-subtrees]
    \label{def:wl_subtree}
    The depth-$L$ WL-subtree $\gT_L(v)$ of vertex $v$ is a rooted tree where: (i) the root represents $v$; (ii) each non-leaf node $u$ has children corresponding to its neighbors $N_G(u)$; and (iii) leaf nodes at depth $L$ are labeled with initial vertex features.
    A graph $G$ induces a set of WL-subtrees, one for each vertex: $\{\gT_L(v) \mid v \in V\}$.
\end{definition}

\begin{lemma}[WL-subtrees Characterize MPNN Expressivity]
    \label{lemma:wl_substree_expressivity}
    Two vertices receive distinct MPNN representations iff their WL-subtrees are non-isomorphic. Similarly, two graphs are distinguishable iff they induce distinct sets of WL-subtrees.
\end{lemma}

Building on the $k$-WL update rule (Equation~\ref{eq:kwl_gnns}), we extend WL-subtrees to higher order:

\begin{definition}[$k$-WL-subtrees]
    \label{def:kfwl_subtree}
    For a graph $G$ and a $k$-tuple $\vv \in V(G)^{k}$, the depth-$L$ $k$-WL-subtree $\gF^k_L(\vv)$ is a rooted tree that records the full computational history of $\vv$ under $k$-WL. The root is labeled by $\vv$.
    Each internal node labeled by a $k$-tuple $\vu$ has children obtained by replacing each position $i \in \{1,\dots,k\}$ with every vertex $t \in V(G)$, yielding child tuples $\vu_{t/i}$, in direct correspondence with the $k$-way aggregation in Equation~\ref{eq:kwl_gnns}.
    Leaf nodes at depth $L$ are labeled with the initial features of the corresponding $k$-tuple.
    A graph $G$ thus induces a collection of $k$-WL-subtrees, one for each $k$-tuple: $\{\gF^k_L(\vv) \mid \vv \in V(G)^{k}\}$.
\end{definition}


\begin{lemma}[$k$-WL-subtrees Characterize $k$-WL Expressivity]
    \label{lemma:kfwl_subtree_expressivity}
    Two $k$-tuples yield distinct $k$-WL representations iff their $k$-WL-subtrees differ in structure or leaf labels. Consequently, two graphs are distinguishable by $k$-WL iff they produce different sets of $k$-WL-subtrees over all $k$-tuples.
\end{lemma}

As illustrated in Figure~\ref{fig:WL_subtree_and_kfwl_subtree}, while vertices $2$ and $2'$ have isomorphic WL-subtrees and thus identical MPNN embeddings, the 3-WL-subtrees of tuples $(1,2,3)$ and $(1',2',3')$ are non-isomorphic---enabling 3-WL to distinguish graphs indistinguishable by 1-WL. This shows that $k$-WL expressivity reduces precisely to the isomorphism of $k$-WL-subtrees.

\begin{figure}[htbp]
    \centering
    \includegraphics[width=1.0\columnwidth]{figures/wl_subtrees_and_kfwl_subtrees.pdf}
    \caption{\label{fig:WL_subtree_and_kfwl_subtree}\textbf{Illustration of WL-Subtrees and 3-WL-Subtrees.}
        (a) Three pairwise non-isomorphic graphs for WL-subtree and 3-WL-subtree construction.
        (b) \textit{WL-subtrees:} WL-subtrees encode iterative neighborhood aggregation. Vertices $0$ and $2$ have distinct WL-subtrees and thus distinct representations. Vertices $2$ and $2'$ share isomorphic WL-subtrees despite differing graphs, forcing identical MPNN embeddings.
        (c) \textit{3-WL-subtrees:} Operating on vertex triples, 3-WL builds richer histories. Tuples $(1,2,3)$ and $(1',2',3')$ yield non-isomorphic 3-WL-subtrees (differences in dashed boxes), enabling discrimination where 1-WL fails.
    }
\end{figure}

\subsection{A Simplicial Abstraction of $k$-WL-Subtrees}
Because $k$-tuples allow repeated vertices, distinct tuples may correspond to the same simplex, introducing structural redundancy that complicates theoretical analysis of $k$-WL-subtrees. 
We resolve this by introducing \textit{simplicial $k$-trees}, a streamlined framework that maps tuples sharing identical vertex sets to a single simplex. 
This abstraction eliminates redundancy and provides a cleaner foundation for characterizing $k$-WL expressivity.


We begin by classifying the simplex-to-simplex relationships induced by parent-child $k$-tuples in a $k$-WL-subtree.
Consider a $k$-tuple $\vv = (v_1, \dots, v_k) \in V(G)^{k}$ (possibly containing repeated vertices). 
During message passing, $\vv$ aggregates from $k$ groups: the $i$-th group comprises all tuples $\{\vv_{t/i} \mid t \in V(G)\}$ for $i \in \{1,\dots,k\}$.
Since the same aggregation structure applies to each group, we focus on a single representative group without loss of generality.
Let $\sigma$ denote the simplex corresponding to $\vv$, and $\sigma_{t/i}$ the simplex corresponding to $\vv_{t/i}$.
The relationship between $\sigma$ and $\sigma_{t/i}$ falls into four cases: 
\begin{enumerate}[wide, labelwidth=!, labelindent=*, nosep, leftmargin=*]
    \item[\textbf{Case 1} ($t=v_i$):] $\sigma_{t/i} = \sigma$;
    \item[\textbf{Case 2} ($t \neq v_i$, $t\in \sigma$, $v_i$ unique in $\vv$):] $\sigma_{t/i}=\sigma \setminus \{v_i\}$, thus $\sigma_{t/i}$ is an immediate face of $\sigma$;
    \item[\textbf{Case 3} ($t \neq v_i$, $t \notin \sigma$, $v_i$ unique in $\vv$):] let $\tau:= \sigma_{t/i} \cap \sigma$, then $\sigma_{t/i}=\tau \cup \{t\}$ and $\sigma=\tau \cup \{v_i\}$, so both $\sigma_{t/i}$ and $\sigma$ are immediate cofaces of their shared face $\tau$;
    \item[\textbf{Case 4} ($t \neq v_i$, $t \notin \sigma$, $v_i$ not unique in $\vv$):] $\sigma_{t/i}=\sigma \cup \{t\}$, thus $\sigma_{t/i}$ is an immediate coface of $\sigma$.
\end{enumerate}

Crucially, Case 3 captures messages from simplices sharing a common face with $\sigma$. This can be decomposed into a two-step pathway: $\sigma$ receives a message from the shared face $\tau$, which in turn receives a message from $\sigma_{t/i}$.
Consequently, all message-passing pathways in a $k$-WL-subtree reduce to repeated face and coface operations.
A depth-$L$ $k$-WL-subtree thus encodes at most $2L$ distinct face-coface pathways, establishing an upper bound on the structural patterns $k$-WL can capture.

Based on the classification, we define simplicial $k$-trees as an abstraction of $k$-WL-subtrees:
\begin{definition}[Simplicial $k$-Trees]
    \label{def:simplicial_ktrees}
    A depth-$2L$ simplicial $k$-tree $\gS^k_{2L}(\sigma)$ is a rooted tree where the root represents a simplex $\sigma \in \gK(G)^{(k{-}1)}$ of dimension $d \le k-1$. Its structure is defined recursively by the following message-passing pathways:
    \begin{itemize}[wide, labelwidth=!, labelindent=*, nosep, leftmargin=*]
        \item \textbf{Face:} If $d > 0$, the node $\sigma$ has children corresponding to each of its immediate faces $\{\tau_1, \dots, \tau_{d}\}$.
        \item \textbf{Coface:} If $d < k-1$, the node $\sigma$ has children corresponding to all its immediate cofaces $\{\varrho \in \gK(G)^{(k{-}1)} \mid \sigma \subseteq \varrho \text{ and } \dim(\varrho) = d+1\}$.
    \end{itemize}
    Leaf nodes at depth $2L$ are labeled with the initial features of the corresponding simplex. A graph $G$ induces a set of simplicial $k$-trees $\{\gS^k_{2L}(\sigma) \mid \sigma \in \gK(G)^{(k{-}1)}\}$.
\end{definition}

\begin{lemma}[Simplicial $k$-Trees Characterize $k$-WL Expressivity]
    \label{lemma:simplicial_ktree_expressivity}
    Two $k$-tuples receive distinct $k$-WL representations iff their corresponding simplicial $k$-trees are non-isomorphic. Consequently, two graphs are distinguishable by $k$-WL iff they induce distinct sets of simplicial $k$-trees.
\end{lemma}

\subsection{Complete Expressivity of $k$-WL GNNs on Graphs without $k$-Cohesive Sets}

Let $\mathcal{G}_{k}$ denote the class of graphs that do not contain any $k$-cohesive sets. We establish that $k$-WL GIN achieves complete expressivity over $\mathcal{G}_{k}$ with respect to both graph isomorphism discrimination and subgraph counting. This expressivity is formalized in the following two theorems.

\begin{theorem}[Isomorphism Discrimination]
\label{thm:isomorphism-discrimination}
Let $G_1, G_2 \in \mathcal{G}_{k}$. If $G_1 \not\cong G_2$, then there exists a $k$-WL GIN that distinguishes $G_1$ from $G_2$.
\end{theorem}

\begin{theorem}[Subgraph Counting Expressivity]
\label{thm:subgraph-expressivity}
For any pattern graph $Q$, any graph $G \in \mathcal{G}_{k}$, and any vertex $v \in V(G)$, there exists a $k$-WL GIN that computes a vertex representation $\mathbf{z}^{k}_{v}$ from which the local subgraph count can be uniquely determined. Formally, there exists a function $f: \mathbb{R}^d \to \mathbb{N}$ such that 
\[
f(\mathbf{z}^{k}_{v}) = \mathrm{iso}(Q, G, v),
\]
where $\mathrm{iso}(Q, G, v)$ denotes the number of subgraphs of $G$ isomorphic to $Q$ that contain vertex $v$.
\end{theorem}

The proofs of these theorems rely on the following structural correspondence lemma.

\begin{lemma}[Bijection between Maximal Simplex Networks and Simplicial $k$-Trees]
\label{lem:maximal-simplex-network-simplicial-k-trees}
Let $G \in \mathcal{G}_{k}$. Then there exists a bijection between the maximal simplex networks of $G$ and its simplicial $k$-tree set. More precisely:
Given the maximal simplex network of $G$, one can uniquely construct the simplicial $k$-tree set of $G$.
Conversely, given the simplicial $k$-tree set of $G$, one can uniquely construct its maximal simplex network.
\end{lemma}

As a result, we have
\begin{corollary}
    For two graphs $G_1, G_2 \in \mathcal{G}_{k}$, if $G_1 \not\cong G_2$, then their simplicial $k$-tree sets are different.
\end{corollary}

\subsection{Bottleneck: Fundamental Expressivity Limits from $k$-Cohesive Sets}

We prove this by constructing a family of graphs that all have the same simplicial $k$-tree set, but different $k$-cohesive sets.
