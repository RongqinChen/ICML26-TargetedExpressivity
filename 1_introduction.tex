\section{Introduction}
Graph Neural Networks (GNNs) have established themselves as the preeminent framework for deep learning on graph-structured data, with pervasive applications in bioinformatics, transportation networks, and recommendation systems. Central to the theoretical foundation of GNNs is the concept of \emph{expressive power}, traditionally characterized by the ability to distinguish non-isomorphic graphs. Recent advancements have expanded this characterization to encompass the detection of topological invariants and higher-order substructures, such as cliques and cycles. While standard message-passing GNNs—exemplified by GCN~\citep{kipf2017semisupervised}, GAT~\citep{velickovic2018graph}, and GIN~\citep{xu2019powerful}—offer substantial practical utility, they are fundamentally upper-bounded in expressivity by the 1-dimensional Weisfeiler–Lehman (1-WL) isomorphism test. Higher-order extensions, based on the $k$-WL or Folklore WL ($k$-FWL) hierarchies~\citep{morris2019weisfeiler}, achieve strictly superior discriminative capacity by operating on $k$-tuples of vertices. However, this theoretical gain comes at the cost of computational and memory requirements that scale combinatorially with $k$, posing a significant \emph{expressivity--efficiency trade-off} that restricts the scalability of such models.

This tension between representational capacity and computational tractability raises two fundamental research questions: (1)~\emph{Can we characterize the graph families for which a low-order $k$-FWL GNN is provably sufficient for discriminative completeness?} and (2)~\emph{Is it possible to circumvent the global overhead of high-order computations by restricting them to specific local regions?} Existing literature provides only limited insights; for instance, 2-FWL is known to be necessary for identifying biconnected components~\citep{zhang2023rethinking,chen2025connectivity}, while 3-FWL is required for 4-clique detection~\citep{huang2023boosting}. A general characterization remains elusive due to the inherent complexity of the graph isomorphism problem
\footnote{Graph isomorphism is a classic problem whose complexity (P vs. NP-complete) remains unresolved.}.
Consequently, this work pursues two analytically tractable objectives: (1)~\textbf{identifying broad structural classes for which low-order $k$-FWL is sufficient}, and (2)~\textbf{developing a principled pruning mechanism that discards redundant high-order interactions without sacrificing the overall expressive power}.

We propose a novel theoretical framework that reinterprets the $k$-FWL mechanism through the lens of abstract simplicial complexes.
In this formulation, the input graph $G$ is embedded into a complete simplicial complex $\gK(G)$, where every subset of $d{+}1$ vertices of $G$ forms an abstract $d$-simplex of $\gK(G)$.
Our key insight is that the message-passing operation in $k$-FWL can be viewed as an aggregation process over the $k$-skeleton of $\gK(G)$.
This topological perspective reveals that $(k{-}1)$-FWL models are less expressive precisely because they are confined to the $(k{-}1)$-skeleton, thereby neglecting the topological information encoded in higher-order $k$-simplices. Furthermore, this abstraction allows us to identify \emph{reducible} simplices—higher-order interactions whose removal does not degrade the discriminative capacity of the $k$-FWL algorithm on $G$.

Leveraging this framework, we formalize the identification and elimination of these redundant structures. We show that if all $k$-simplices in $\gK(G)$ are reducible, the $k$-skeleton effectively collapses to the $(k{-}1)$-skeleton, rendering $(k{-}1)$-FWL equivalent to $k$-FWL in terms of expressivity on $G$.
To operationalize this, we introduce \emph{$c$-cohesiveness} as a novel structural invariant: a vertex set $S$ ($|S| \geq c{+}1$) is defined as $c$-cohesive if for every pair of distinct vertices $u, v \in S$, $\kappa_G(u,v) \geq c$.
Crucially, these paths may traverse vertices external to $S$; consequently, the subgraph induced by $S$, denoted $G[S]$, is not required to be $c$-connected.

Our principal theoretical contributions are summarized as follows. First, we prove that for any graph family lacking $(k{+}1)$-cohesive vertex sets, $k$-FWL provides a complete discriminative basis. Second, we introduce the \emph{pruned $k$-FWL GNN}, an architecture that operates selectively on the sub-complex of $c$-cohesive simplices (for $c \leq k$), and rigorously demonstrate that it preserves the full expressive power of the unpruned $k$-FWL. Third, we establish that non-cohesive simplices are universally reducible across the $k$-FWL hierarchy. Together, these results provide a rigorous characterization of minimal sufficient expressivity and offer a theoretically grounded strategy for navigating the expressivity--efficiency frontier in higher-order graph neural architectures.

The manuscript is organized as follows.
Section~\ref{sec:preliminaries} provides essential background.
Section~\ref{sec:reinterpreting_kfwl} reinterprets $k$-FWL GNNs through the lens of simplicial complexes.
Section~\ref{sec:sufficiency_redundancy} establishes conditions under which low-order $k$-FWL is sufficient and identifies redundant higher-order interactions.
Section~\ref{sec:connectivity_adaptive_hierarchical_gnns} proposes a connectivity-aware pruning mechanism for $k$-FWL GNNs.
Section~\ref{sec:experiments} presents empirical evaluations, and
Section~\ref{sec:conclusion} discusses implications and future directions.
Due to space constraints, we defer extended background to Appendix~\ref{sec:detailed_background} and a comprehensive survey of related work on GNN expressivity and higher-order architectures to Appendix~\ref{sec:related_work}.
Appendix~\ref{sec:counterexamples} includes counterexamples confirming that $\theta$-cohesive $\theta$-simplices generally require $\theta$-FWL computation and are thus irreducible, while non-cohesive simplices are safely pruned.

A GNN gains expressive power when it can distinguish intricate substructures such as cliques and cycles.
However, the link between its message-passing dynamics and the underlying topology of the input graph remains opaque, obscuring our understanding of expressivity, redundancy, and generalization.
We make this link explicit through the \emph{simplicial representation} of a graph $G$: an abstract simplicial complex $\Delta(G)$ in which each $d$-simplex corresponds to a $(d+1)$-vertex motif with strong internal connectivity that is provably hard for GNNs to distinguish.
In this representation, face--coface relations encode refinements of vertex tuples under higher-order message passing, yielding a topological perspective on how GNNs process and differentiate graph structure.

