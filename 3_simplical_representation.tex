\section{Simplicial Representation of Graphs}
\label{sec:simplical_representation}

Higher-order GNNs generalize classical message passing by operating on vertex tuples rather than on single vertices or edges, which allows them to distinguish patterns such as cliques and cycles.
However, the relation between these tuple-based mechanisms and the intrinsic topology of the underlying graph is not immediate.
We make this relation explicit via the \emph{simplicial representation} of a graph $G$: an abstract simplicial complex $\Delta(G)$ in which each $d$-simplex corresponds to a $(d+1)$-vertex motif that exhibit strong interactions. In this view, face--coface relations track refinements of vertex tuples in higher-order message passing, providing a topological perspective on both the expressive power and the redundancy of higher-order GNNs.

\subsection{$c$-Cohesive Sets}
To formalize $(d+1)$-vertex motifs that exhibit strong interactions,
we introduce the concept of \emph{$c$-cohesive sets}:
\begin{definition}[$c$-Cohesive Set]
    \label{def:c-cohesive}
    A vertex set $S \subseteq V(G)$ is \emph{$c$-cohesive} for $c \in \mathbb{Z}_{\ge 0}$ if either $c = 0$ and $|S| = 1$, or $c \ge 1$, $|S| \ge c+1$, and the local connectivity satisfies $\kappa_G(u,v) \ge c$ for all distinct $u,v \in S$.
\end{definition}

By Menger's theorem, $S$ is $c$-cohesive exactly when removing any set of at most $c-1$ vertices leaves all vertices of $S$ in the same connected component. Crucially, $c$-cohesiveness depends on how $S$ sits inside $G$: the internally disjoint paths connecting vertices of $S$ may traverse vertices outside $S$.

\begin{remark}
    \label{rem:cohesive_properties}
    Cohesiveness is monotone in $c$: if $S$ is $c$-cohesive, then it is also $c'$-cohesive for every $c' \le c$. Moreover, any \emph{minimal} $c$-cohesive set has exactly $c+1$ vertices, and each of its proper subsets is a minimal $c'$-cohesive set for some $c' < c$.
\end{remark}

\begin{lemma}[Union Property]
    \label{lem:union_cohesive}
    If two $c$-cohesive sets $S_1, S_2$ share at least $c$ vertices, then their union $S_1 \cup S_2$ is also $c$-cohesive.
\end{lemma}

Minimal $c$-cohesive sets thus act as atomic units, from which larger cohesive regions arise via overlapping unions.


\subsection{Cohesive Simplicial Complex}

To organize cohesive sets combinatorially, we introduce the cohesive simplicial complex:
\begin{definition}[Cohesive Simplicial Complex]
    \label{def:cohesive-complex}
    The \emph{cohesive simplicial complex} $\Delta(G)$ is the abstract simplicial complex whose simplices are precisely the minimal $c$-cohesive sets of $G$ for all $c \ge 0$. To retain the underlying graph structure, each $1$-simplex carries a label indicating whether its two vertices form an edge of $G$.
\end{definition}

By Remark~\ref{rem:cohesive_properties}, $\Delta(G)$ is downward-closed and hence satisfies the axioms of an abstract simplicial complex. Its nested simplices encode a multi-scale hierarchy of cohesive regions in $G$.

\subsection{Tree Representation via Facets}

The global organization of $\Delta(G)$ is controlled by how its facets intersect along minimal separators in $G$.

\begin{lemma}[Separator Role of Intersections]
    \label{lem:intersection-separator}
    The intersection of any two distinct maximal simplices in $\Delta(G)$ is a minimal vertex separator of $G$.
\end{lemma}

Together, Lemmas~\ref{lem:union_cohesive} and~\ref{lem:intersection-separator} yield a canonical tree that hierarchically organizes cohesive regions, which is defined as follows:
\begin{definition}[Maximal Simplex Network]
    \label{def:maximal_simplex_network}
    The \emph{maximal simplex network} of $G$ is a bipartite graph with two vertex classes: maximal simplices of $\Delta(G)$ and their pairwise intersections (minimal separators). A maximal simplex connects to a separator if that separator is a proper face of the simplex.
\end{definition}

\begin{lemma}[Tree Structure]
    \label{lem:maximal_simplex_network_acyclic}
    The maximal simplex network of any graph $G$ is acyclic (a forest, and a tree if $G$ is connected) and is a structural invariant of $G$.
\end{lemma}

Thus, the maximal simplex network gives a compact, tree-structured summary of how cohesive substructures in $G$ are arranged, which is well-suited for theoretical analysis and applications such as graph isomorphism testing.
