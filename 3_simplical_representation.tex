\section{Simplicial Representation of Graphs}
To capture the higher-order connectivity that demands higher-order GNNs,
we introduce a multi-scale hierarchy of \emph{$c$-cohesive vertex sets}---subsets connected via at least $c$ vertex-disjoint paths.
These sets form a \emph{cohesive simplicial complex}, a topological scaffold encoding multi-scale regions with robust connectivity.

\subsection{$c$-Cohesive Sets}
We extend vertex connectivity from pairs to sets.
\begin{definition}[$c$-Cohesive Set]
    \label{def:c-cohesive}
    A non-empty vertex set $S \subseteq V(G)$ is \emph{$c$-cohesive} for integer $c \ge 0$ if either:
    \begin{enumerate}[leftmargin=*,nosep]
        \item $c = 0$ and $|S| = 1$, or
        \item $c \ge 1$, $|S| \ge c+1$, and $\kappa_G(u,v) \ge c$ for all distinct $u,v \in S$.
    \end{enumerate}
\end{definition}

By Menger's theorem, condition (2) ensures every vertex pair in $S$ remains connected after the removal of any $c-1$ vertices from $G$. Crucially, the vertex-disjoint paths that witness this robustness may use vertices outside $S$; thus, $c$-cohesiveness is a property of the set's \emph{embedding in the full graph}, not merely of its induced subgraph $G[S]$.

A $c$-cohesive set is \emph{maximal} (called a \emph{$c$-block}) if no proper superset is $c$-cohesive, and \emph{minimal} if none of its proper subsets is $c$-cohesive. Key properties follow directly:
\begin{itemize}[leftmargin=*,nosep]
    \item \textbf{Monotonicity:} If $S$ is $c$-cohesive, it is also $c'$-cohesive for all $0 \le c' \le c$.
    \item \textbf{Size of Minimal Sets:} Every minimal $c$-cohesive set has exactly $c+1$ vertices.
    \item \textbf{Downward Closure:} If $S$ is minimal $c$-cohesive and $c' < c$, then some $(c'+1)$-subset of $S$ is minimal $c'$-cohesive.
\end{itemize}

Minimal cohesive sets are the basic units of robust structure.
They can aggregate into larger cohesive regions through overlap:

\begin{lemma}
    \label{lem:union_cohesive}
    Let $S_1, S_2$ be $c$-cohesive sets in $G$. If $|S_1 \cap S_2| \ge c$, then $S_1 \cup S_2$ is also $c$-cohesive.
\end{lemma}

\subsection{The Cohesive Simplicial Complex}
We now formalize the aggregation of all minimal cohesive sets into a unified combinatorial-topological structure.

\begin{definition}[Cohesive Simplicial Complex]
    \label{def:cohesive-complex}
    The \emph{cohesive simplicial complex} $\Delta(G)$ of a graph $G$ is the abstract simplicial complex defined by
    \begin{align*}
        \Delta(G)
            &= \bigl\{\sigma \subseteq V(G) : \sigma \neq \emptyset \text{ and } \sigma \text{ is minimal $c$-cohesive} \\
            &\qquad\text{for some } c \ge 0\bigr\},
    \end{align*}
    where a simplex $\sigma \in \Delta(G)$ is included along with all its non-empty subsets.
\end{definition}

\begin{remark}
    The closure property required for an abstract simplicial complex follows immediately from the monotonicity and downward-closure of $c$-cohesiveness (cf.\ Section~\ref{def:c-cohesive}): if $\sigma$ is minimal $c$-cohesive, then every non-empty subset $\tau \subseteq \sigma$ is minimal $c'$-cohesive for some $c' \le c$.
\end{remark}

To preserve the original graph structure, we equip $\Delta(G)$ with an edge-indicator labeling on any $1$-simplex $\{u,v\}$:
\[
    \ell_G(\{u,v\}) = \mathds{1}[(u,v) \in E(G)].
\]

\paragraph{Geometric and Topological Interpretation.}
For a simplex $\sigma \in \Delta(G)$, its dimension satisfies $\dim(\sigma) = |\sigma| - 1 = c$, where $c$ is the cohesion level of $\sigma$.
By Lemma~\ref{lem:union_cohesive}, the union of overlapping simplices $\sigma_1, \sigma_2$ with $|\sigma_1 \cap \sigma_2| \ge c$ forms a larger $c$-cohesive region.
Consequently, $\Delta(G)$ encodes a multi-scale filtration of connectivity: the $1$-skeleton $\Delta^{(1)}(G)$ captures pairwise adjacency, while each higher-dimensional simplex ($\dim \ge 2$) represents a globally robust interaction pattern sustained by multiple vertex-disjoint paths---structural motifs that are inherently invisible to message-passing GNNs operating solely on the $1$-skeleton.

\paragraph{Graph Isomorphism and Expressivity.}
The transformation $G \mapsto (\Delta(G), \ell_G)$ is information-preserving in the sense that it establishes a bijective correspondence between graph isomorphism classes and labeled simplicial complex isomorphism classes. This reversibility is formalized below.

\begin{proposition}[Isomorphism Equivalence]
    \label{prop:iso-equivalence}
    Let $G$ and $H$ be graphs, and denote by $\Delta(G), \Delta(H)$ their cohesive simplicial complexes equipped with edge-indicator labelings $\ell_G, \ell_H$ on $1$-simplices. Then
    \[
        G \cong H \quad \Longleftrightarrow \quad (\Delta(G), \ell_G) \cong (\Delta(H), \ell_H).
    \]
\end{proposition}

\begin{corollary}[Expressivity Criterion]
    \label{cor:expressivity-criterion}
    A GNN can distinguish non-isomorphic graphs $G \not\cong H$ if it can distinguish the corresponding labeled simplicial complexes $(\Delta(G), \ell_G) \not\cong (\Delta(H), \ell_H)$ by learning to:
    \begin{enumerate}[label=(\roman*),nosep]
        \item differentiate simplices embedded in distinct local neighborhoods, and
        \item propagate information about the edge labels between simplices across the complex.
    \end{enumerate}
\end{corollary}

This result underscores that extending message-passing from edges to higher-dimensional simplices is sufficient for capturing the higher-order structural patterns encoded by $\Delta(G)$.

Note that this criterion is neither unique nor optimal: a GNN may distinguish certain non-isomorphic graphs without satisfying it. 
Nevertheless, it is still crucial for understanding GNN expressivity for several reasons:
(i) it closely aligns with the message-passing paradigms of $k$-WL/$k$-FWL GNNs, enabling a proof of their complete expressivity on graphs with bounded cohesion;
(ii) it provides a principled way to construct counterexamples where specific GNN architectures fail to distinguish non-isomorphic graphs;
(iii) it forms a foundation for identifying redundant computations in GNNs and designing more efficient algorithms; and
(iv) it offers a framework to compare the expressivity of Local $k$-FGNNs with other GNN models, addressing an open problem raised in~\cite{zhang2024beyond}.
