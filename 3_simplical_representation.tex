\section{Simplicial Representation of Graphs}
To capture higher-order connectivity patterns beyond pairwise edges, we introduce a hierarchy of \emph{$c$-cohesive vertex sets} based on vertex connectivity. These sets serve as building blocks for a simplicial complex encoding structural cohesion across multiple scales. We define the \emph{cohesive simplicial complex}, a topological representation that generalizes graph connectivity to higher dimensions while preserving the original graph structure.

\subsection{$c$-Cohesive Sets}
We formalize tightly connected vertex subsets extending the notion of local connectivity.

\begin{definition}[$c$-Cohesive Set]
    \label{def:c-cohesive}
    A non-empty vertex set $S \subseteq V(G)$ is \emph{$c$-cohesive} for integer $c \ge 0$ if either:
    \begin{enumerate}[leftmargin=*,nosep]
        \item $c = 0$ and $|S| = 1$ (singletons are trivially cohesive), or
        \item $c \ge 1$, $|S| \ge c+1$, and $\kappa_G(u,v) \ge c$ for all distinct $u, v \in S$.
    \end{enumerate}
\end{definition}

By Menger's theorem, condition (2) requires that every vertex pair in $S$ remains mutually reachable after removing any $c-1$ vertices from $G$. The vertex-disjoint paths witnessing this robustness may traverse vertices outside $S$; thus, $c$-cohesiveness depends on the global embedding of $S$ in $G$, not solely on the induced subgraph $G[S]$.

A $c$-cohesive set is (inclusion-wise) \emph{minimal} if no proper non-empty subset is $c$-cohesive. Definition~\ref{def:c-cohesive} implies: (i) cohesiveness is monotone—if $S$ is $c$-cohesive, then $S$ is $c'$-cohesive for all $c' \le c$; (ii) any minimal $c$-cohesive set has exactly $c+1$ vertices; (iii) every $(c'{+}1)$-subset of such a minimal set is minimal $c'$-cohesive for some $c' < c$.

Minimal cohesive sets thus serve as atomic motifs of structural resilience, while larger cohesive regions assemble from their overlaps via the following property.

\begin{lemma}
    \label{lem:union_cohesive}
    Let $S_1, S_2$ be $c$-cohesive sets in $G$. If $|S_1 \cap S_2| \ge c$, then $S_1 \cup S_2$ is also $c$-cohesive.
\end{lemma}

\subsection{The Cohesive Simplicial Complex}
We consolidate all cohesive motifs into a simplicial complex organizing them across scales.

\begin{definition}[Cohesive Simplicial Complex]
    \label{def:cohesive-complex}
    The \emph{cohesive simplicial complex} $\Delta(G)$ is the abstract simplicial complex whose simplices are all non-empty subsets of $V(G)$ that are $c$-cohesive for some $c \ge 0$.
\end{definition}

Closure under non-empty subsets follows from monotonicity of cohesiveness, satisfying the axioms of an abstract simplicial complex. For simplex $S \in \Delta(G)$, its dimension is $\dim(S) = |S| - 1$, and its \emph{cohesion level} is the largest $c$ such that $S$ is $c$-cohesive.

A key property is the structure of its $1$-skeleton. Since any two vertices $u,v$ in the same connected component of $G$ satisfy $\kappa_G(u,v) \ge 1$, they form a $1$-cohesive set. Consequently, $\Delta(G)^{(1)}$ is a complete graph on each connected component of $G$. To preserve the original adjacency structure, we annotate each $1$-simplex $\{u,v\}$ with binary label $\ell(\{u,v\}) = \mathds{1}[(u,v) \in E(G)]$.

This construction yields a multi-scale topological representation of $G$ in which higher-dimensional simplices encode dense, resilient interaction patterns invisible to standard message-passing GNNs operating solely on $E(G)$.
