\section{Simplicial Representation of Graphs}
Graphs contain higher-order interactions—dense, multi-way connectivity patterns that challenge the expressive power of GNNs. To capture these structures, we introduce a hierarchy of \emph{cohesive vertex sets} grounded in vertex connectivity and organize them into a \emph{simplicial complex} that reflects the multi-scale architecture of a graph. This complex yields a canonical, tree-structured invariant: the \emph{simplex network}, which encodes the full face--coface inclusion hierarchy. The simplex network provides expressive, interpretable representations for graph comparison and substructure enumeration, supports rigorous analysis of GNN expressivity, and inspires the design of more efficient yet provably expressive architectures.

\subsection{$c$-Cohesive Sets}
We begin by formalizing tightly connected vertex subsets using classical connectivity.
\begin{definition}[$c$-Cohesive Set]
    \label{def:c-cohesive}
    A vertex set $S \subseteq V(G)$ is \emph{$c$-cohesive} for $c \in \mathbb{Z}_{\ge 0}$ if either (i) $c = 0$ and $|S| = 1$, or (ii) $c \ge 1$, $|S| \ge c+1$, and $\kappa_G(u,v) \ge c$ for all distinct $u,v \in S$, where $\kappa_G(u,v)$ denotes the local vertex connectivity between $u$ and $v$ in $G$.
\end{definition}
By Menger’s theorem, this means that removing any $c-1$ vertices from $G$ leaves all vertices of $S$ in the same connected component. Crucially, the witnessing paths may traverse vertices outside $S$, so $c$-cohesiveness depends on the global embedding of $S$ in $G$, not just its induced subgraph.

\begin{remark}
    \label{rem:cohesive_properties}
    Cohesiveness is monotone in $c$: if $S$ is $c$-cohesive, then it is $c'$-cohesive for every integer $c'$ with $0 \le c' \le c$. A $c$-cohesive set is \emph{minimal} if and only if it has exactly $c+1$ vertices. Moreover, every non-empty proper subset of such a set with size $c'+1$ is a minimal $c'$-cohesive set for some $c' < c$.
\end{remark}

These minimal sets serve as atomic motifs of cohesion. Larger cohesive regions emerge via overlap:

\begin{lemma}
    \label{lem:union_cohesive}
    If two $c$-cohesive sets $S_1$ and $S_2$ satisfy $|S_1 \cap S_2| \ge c$, then $S_1 \cup S_2$ is also $c$-cohesive.
\end{lemma}

\subsection{The Cohesive Simplicial Complex}

We now organize these motifs into a simplicial complex.

\begin{definition}[Cohesive Simplicial Complex]
    \label{def:cohesive-complex}
    The \emph{cohesive simplicial complex} $\Delta(G)$ is the abstract simplicial complex whose simplices are precisely the minimal $c$-cohesive sets of $G$ for all $c \ge 0$. To preserve the underlying graph structure, each $1$-simplex $\{u,v\}$ is labeled to indicate whether $(u,v) \in E(G)$.
\end{definition}

By Remark~\ref{rem:cohesive_properties}, $\Delta(G)$ is closed under taking non-empty subsets, satisfying the axioms of an abstract simplicial complex. Its dimension reflects the highest order of cohesion present in $G$, and inclusion among simplices encodes refinement across scales.

\subsection{From Facets to the Simplex Network}

The maximal simplices (facets) of $\Delta(G)$ reveal how cohesive cores intersect. Their overlaps correspond to fundamental graph separators:

\begin{lemma}
    \label{lem:intersection-separator}
    The intersection of any two distinct facets in $\Delta(G)$ is a minimal vertex separator of $G$.
\end{lemma}

This leads to an intermediate bipartite representation:

\begin{definition}[Facet Network]
    \label{def:facet_network}
    The \emph{facet network} of $G$ is a bipartite graph with facet nodes (maximal simplices of $\Delta(G)$) and separator nodes (their pairwise intersections), connected when a separator is a face of a facet.
\end{definition}

The facet network is acyclic and invariant under graph isomorphism, but it omits non-maximal simplices. 
Thus, we refine it to include all simplices and add edges to encode the face--coface relation.

\begin{definition}[Simplex Network]
    \label{def:simplex_network}
    The \emph{simplex network} $\Upsilon(G)$ is the graph whose vertices are all simplices of $\Delta(G)$, 
    and where two simplices $\tau$ and $\sigma$ are adjacent if one is a maximal proper face of the other—that is, 
    $\tau \subsetneq \sigma$ and $\dim(\sigma) = \dim(\tau) + 1$, or vice versa.
\end{definition}

$\Upsilon(G)$ can be constructed from the facet network by (1) expanding each facet--separator edge into a maximal chain of nested simplices, and (2) recursively adding all missing cover relations. Because every minimal separator remains a cut node, the tree-like structure is preserved:

\begin{lemma}
    \label{lem:simplex_network_acyclic}
    The simplex network $\Upsilon(G)$ is acyclic (a tree if $G$ is connected) and is a structural invariant of $G$.
\end{lemma}

Moreover, it faithfully reflects subgraph structure:

\begin{lemma}
    \label{lem:subgraph-subnetwork-correspondence}
    For any connected subgraph $Q \subseteq G$, the minimal $c$-cohesive sets contained in $V(Q)$ form a connected subcomplex of $\Delta(G)$, inducing a connected subnetwork in $\Upsilon(G)$.
\end{lemma}

Thus, the \emph{simplex network} serves as our primary structural invariant: a compact, hierarchical, and interpretable encoding of higher-order cohesion in $G$, designed to overcome the limitations of standard GNNs and enable both practical graph analytics and theoretical expressivity analysis.

