\section{Simplicial Representation of Graphs}
To capture the higher-order connectivity that demands higher-order GNNs,
we introduce a multi-scale hierarchy of \emph{$c$-cohesive vertex sets}---subsets connected via at least $c$ vertex-disjoint paths.
These sets form a \emph{cohesive simplicial complex}, a topological scaffold encoding multi-scale regions with robust connectivity.

\subsection{$c$-Cohesive Sets}
We extend vertex connectivity from pairs to sets.
\begin{definition}[$c$-Cohesive Set]
    \label{def:c-cohesive}
    A non-empty vertex set $S \subseteq V(G)$ is \emph{$c$-cohesive} for integer $c \ge 0$ if either:
    \begin{enumerate}[leftmargin=*,nosep]
        \item $c = 0$ and $|S| = 1$, or
        \item $c \ge 1$, $|S| \ge c+1$, and $\kappa_G(u,v) \ge c$ for all distinct $u,v \in S$.
    \end{enumerate}
\end{definition}

By Menger's theorem, condition (2) ensures every vertex pair in $S$ remains connected after the removal of any $c-1$ vertices from $G$. Crucially, the vertex-disjoint paths that witness this robustness may use vertices outside $S$; thus, $c$-cohesiveness is a property of the set's \emph{embedding in the full graph}, not merely of its induced subgraph $G[S]$.

A $c$-cohesive set is \emph{maximal} (called a \emph{$c$-block}) if no proper superset is $c$-cohesive, and \emph{minimal} if none of its proper subsets is $c$-cohesive. Key properties follow directly:
\begin{itemize}[leftmargin=*,nosep]
    \item \textbf{Monotonicity:} If $S$ is $c$-cohesive, it is also $c'$-cohesive for all $0 \le c' \le c$.
    \item \textbf{Size of Minimal Sets:} Every minimal $c$-cohesive set has exactly $c+1$ vertices.
    \item \textbf{Downward Closure:} If $S$ is minimal $c$-cohesive and $c' < c$, then some $(c'+1)$-subset of $S$ is minimal $c'$-cohesive.
\end{itemize}

Minimal cohesive sets are the basic units of robust structure.
They can aggregate into larger cohesive regions through overlap:

\begin{lemma}
    \label{lem:union_cohesive}
    Let $S_1, S_2$ be $c$-cohesive sets in $G$. If $|S_1 \cap S_2| \ge c$, then $S_1 \cup S_2$ is also $c$-cohesive.
\end{lemma}

\subsection{The Cohesive Simplicial Complex}
We integrate all minimal cohesive sets into a single combinatorial-topological object.
\begin{definition}[Cohesive Simplicial Complex]
    \label{def:cohesive-complex}
    The \emph{cohesive simplicial complex} $\Delta(G)$ is the abstract simplicial complex whose 
    simplices are all non-empty subsets of $V(G)$ that are \emph{minimal $c$-cohesive} for some $c \ge 0$.
\end{definition}
Closure under taking subsets follows from the monotonicity and downward-closure properties, satisfying the axioms of an abstract simplicial complex.
The original graph is preserved by labeling each $1$-simplex $\{u,v\}$ with $\ell(\{u,v\}) = \mathbb{1}[(u,v) \in E(G)]$.

For a simplex $\sigma$, its dimension $\dim(\sigma)=|\sigma|-1$ equals its cohesion level $c$. The union of overlapping simplices represents a larger cohesive region whose cohesion level is determined by the minimum size of their pairwise intersections.
Thus $\Delta(G)$ provides a multi-scale topological representation: its $1$-skeleton retains local adjacency, while each higher-dimensional simplex ($\dim \ge 2$) encodes a globally robust, multi-vertex interaction pattern invisible to standard GNNs.

\subsection{Graph Invariants via Cohesive Simplicial Complex}

Any connected graph admits a canonical decomposition into a tree of $2$-connected components, known as its \emph{block–cut tree}. 
Each node of this tree is either a \emph{block} (a maximal $2$-connected subgraph, a bridge (or an edge connecting two vertices), or an isolated vertex) or a \emph{cut vertex} (also called an articulation point), and edges encode incidences between blocks and cut vertices. 
A cut vertex is a vertex whose removal strictly increases the number of connected components.
